\documentclass{beamer}
\usetheme{Boadilla}
\usepackage{ifpdf}
\ifpdf
\usepackage[utf8]{inputenc}
\usepackage[T1]{fontenc}
\usepackage[all,pdf,2cell]{xy}\UseAllTwocells\SilentMatrices
\else
\usepackage[all,xdvi,2cell]{xy}\UseAllTwocells\SilentMatrices
\fi

\usepackage{newunicodechar}
\usepackage{cmbright}
\usepackage{mathtools}
\usepackage{amsthm}
\usepackage{amsmath}
\usepackage{amsfonts}

\newtheorem{proposition}[theorem]{Proposition}
\newcommand{\id}{\mathrm{id}}
\newcommand{\obj}{\mathrm{obj}}
\newcommand{\C}{\mathcal{C}}
\newcommand{\D}{\mathcal{D}}
\newcommand{\isleftadjoint}{\dashv}
\newcommand{\VoltG}{{\mathbf{Volt}_Γ}}
\newcommand{\LabG}{{\mathbf{Lab}_Γ}}
\newcommand{\ActLab}{{\mathbf{ActLab}_Γ}}
\newcommand{\KG}{{\mathring{K}(Γ)}}
\newcommand{\K}{\mathring{K}}
\renewcommand{\k}{\mathring{k}}
\newcommand{\lG}{{\ell(Γ)}}
\newcommand{\pb}[3]{{#1}×_{#2}{#3}}
\newcommand{\newcategory}[1]{\expandafter\newcommand\csname #1\endcsname{\mathbf{#1}}}
\newcommand{\Graph}{\mathbf{Graph}}
\newcommand{\Volt}{\mathbf{Volt}}
\newcommand{\Lab}{\mathbf{Lab}}
\newcommand{\Vect}{\mathbf{Vect}}
\newcommand{\FinVect}{\mathbf{FinVect}}
\newcommand{\FinHilb}{\mathbf{FinHilb}}
\newcommand{\RelPos}{\mathbf{RelPos}}
\newcommand{\RelPosInv}{\mathbf{RelPosInv}}
\makeatletter
\def\slashedarrowfill@#1#2#3#4#5{%
  $\m@th\thickmuskip0mu\medmuskip\thickmuskip\thinmuskip\thickmuskip
  \relax#5#1\mkern-7mu%
  \cleaders\hbox{$#5\mkern-2mu#2\mkern-2mu$}\hfill
  \mathclap{#3}\mathclap{#2}%
  \cleaders\hbox{$#5\mkern-2mu#2\mkern-2mu$}\hfill
  \mkern-7mu#4$%
}
\def\rightslashedarrowfill@{%
  \slashedarrowfill@\relbar\relbar\mapstochar\rightarrow}
\newcommand\xslashedrightarrow[2][]{%
  \ext@arrow 0055{\rightslashedarrowfill@}{#1}{#2}}
\makeatother
\newcommand{\sto}{\xslashedrightarrow{\hskip 1pc}}

\newunicodechar{α}{\alpha}
\newunicodechar{β}{\beta}
\newunicodechar{λ}{\lambda}
\newunicodechar{Γ}{\Gamma}
\newunicodechar{×}{\times}
\newunicodechar{≤}{\leq}
\newunicodechar{≥}{\geq}

\title[Monotone relations and EAs]{Monotone relations and effect algebras}
\author{Gejza Jenča}
\institute[]{Slovak University of Technology Bratislava}
\date{\today}

\begin{document}
\begin{frame}
\titlepage

\tiny This research is supported by grants VEGA 2/0142/20 and 1/0006/19,
Slovakia and by the Slovak Research and Development Agency under the contracts
APVV-18-0052 and APVV-16-0073.
\end{frame}
\begin{frame}
\frametitle{The definition}
\begin{definition}
Let $P,Q$ be posets. A relation $f\subseteq P\times Q$ is {\em monotone relation
from $P$ to $Q$}
if and only if, for every $p_1,p_2\in P$ and $q_1,q_2\in Q$,
$$
p_2\geq p_1\text{ and } f(p_1,q_2)\text{ and }q_2\geq q_1\text{ imply } f(p_2,q_1)
$$
\end{definition}
We write $f\colon P\sto Q$ for a monotone relation $f$ from $P$ to $Q$.
\begin{example}
Let $P$, $Q$ be posets. Both the universal 
relation $P\times Q\subseteq P\times Q$ and the empty relation $\emptyset\subseteq P\times
Q$ are monotone.
\end{example}
\end{frame}

\begin{frame}
It is instructive to visualize a monotone relation between two disjoint finite
posets $P$ and $Q$ in the following way.
\begin{itemize}
\item Draw the Hasse diagram of $Q$.\pause
\item Draw the Hasse diagram of $P$ over the diagram of $Q$.\pause
\item Draw some additional lines between elements of $P$ and elements of $Q$,\pause
so that the resulting picture is a Hasse diagram of a poset $C$.\pause
\item This poset then determines a monotone relation $f_C\subseteq P\times Q$ 
given by the rule $f_C(p,q)$ if and only if $q\leq_C p$.
\end{itemize}
\end{frame}
\begin{frame}
\begin{center}
\includegraphics{PQ}
\includegraphics{cograph}
\end{center}
\end{frame}
\begin{frame}
For every monotone mapping $f\colon P\to Q$, there is a monotone relation
$\widehat f\colon P\sto Q$ given by
$$
\widehat f(p,q)\Leftrightarrow f(p)\geq q
$$
\end{frame}

\begin{frame}
Let $P,Q,R$ be posets, let $f\colon P\sto Q$ and $g\colon Q\sto R$ be
monotone relations. The composite relation $g\circ f\subseteq|P|\times|R|$ is given
by the rule
$$
(g\circ f)(p,r)\text{ if and only if }
	f(p,q)\text{ and }g(q,r)\text{ for some }q\in Q.
$$
It is easy to check that the composite relation of two monotone relations is
monotone and that the operation of composition is associative.
\end{frame}

\begin{frame}
For a poset $P$, the {\em identity monotone relation} is the relation
$\id_P\colon P\sto P$ given by the rule
$$
\id_P(x,y)\Leftrightarrow x\geq y
$$
It is easy to see that for every monotone relation $f\colon P\sto Q$,
$f=\id_Q\circ f=f\circ\id_P$.
\end{frame}

\begin{frame}
The category of posets and monotone relations, denoted by $\RelPos$, is a category
whose objects are posets and morphisms are monotone relations.

\end{frame}
\begin{frame}
The direct product $\otimes$ of posets is a bifunctor from
$\RelPos\times\RelPos$ to $\RelPos$. Indeed, for a pair of monotone relations
$f\colon A\sto B$ and $g\colon C\sto D$ the monotone relation $(f\otimes g)\colon
A\otimes C\sto B\otimes D$ by the rule
$$
(f\otimes g)((a,c),(b,d)) \Leftrightarrow f(a,b)\text{ and }g(c,d)
$$
\end{frame}
\begin{frame}
\begin{itemize}
\item 
Fix a $1$-element poset and call it $\mathbf 1$.
\item
$(\RelPos,\otimes,\mathbf 1)$ is a symmetric
monoidal category
\item For cographs, we have $cog(f\otimes g)=cog(f)\times cog(g)$.
\end{itemize}

\end{frame}
\begin{frame}

A {\em dual object} to an object $A$ of a symmetric monoidal category $(\C,\otimes,I)$ is an object
$A^*$ such that there are morphisms
$\eta_A\colon I\to A^*\otimes A$ and $\epsilon_A\colon A\otimes A^*\to I$ such
that the diagrams
\begin{equation}
\label{diag:compact}
\xymatrix{
A
	\ar[r]^{\rho_A^{-1}}
	\ar[dd]_{\id_A}
&
A\otimes I
	\ar[d]^{\id_A\otimes\eta_A}
\\
~
&
A\otimes A^*\otimes A
	\ar[d]^{\epsilon_A\otimes\id_A}
\\
A
&
I\otimes A
	\ar[l]^{\lambda_A}
}
\qquad
\xymatrix{
A^*
	\ar[r]^{\lambda_{A^*}^{-1}}
	\ar[dd]_{\id_{A*}}
&
I\otimes A^*
	\ar[d]^{\eta_A\otimes\id_{A^*}}
\\
~
&
A^*\otimes A\otimes A^*
	\ar[d]^{\id_{A^*}\otimes\epsilon_A}
\\
A
&
A^*\otimes I
	\ar[l]^{\rho_{A^*}}
}
\end{equation}
commute. The morphisms $\eta_A$ and $\epsilon_A$ are called {\em coevaluation} and
{\em evaluation}, respectively. 
\end{frame}
\begin{frame}
\begin{itemize}
\item If $A^*$ and $A^+$ are dual objects of an object $A$, then $A^*\simeq A^+$.\pause
\item Consider the symmetric monoidal category of vector spaces 
equipped with the tensor product $(\Vect(K),\otimes,K)$. A vector space $V$ has
a dual iff $V$ is finitely dimensional;\pause
\item the dual object of $V$ is then simply the usual linear dual of $V$.\pause
\item Modules over a commutative ring: $M$ has a dual iff $M$ is a finitely generated
projective module.\pause
\item By fixing a chosen dual for each object, taking a dual can be made to a
contravariant functor $*\colon\C\to\C^{op}$.\pause
\item So, for a morphism $f\colon A\to B$ there is a dual morphism $f^*\colon B^*\to
A^*$.\pause
\end{itemize}
\end{frame}


\begin{frame}
\begin{definition}
A symmetric monoidal category is {\em compact closed}
if every of its objects has a dual.
\end{definition}

The category of finite-dimensional vector spaces $\FinVect(K)$ is compact closed.
\end{frame}
\begin{frame}
\begin{theorem}
\begin{itemize}
\item
$(\RelPos,\times,\mathbf 1)$ is a compact closed category. 
\item The dual object of a poset $A$
is the dual poset of $A$.
\item For cographs: $cog(f^*)\simeq (cog(f))^*$.
\end{itemize}
\end{theorem}
\end{frame}
\begin{frame}
\begin{itemize}
\item $\FinVect(\mathbb C)$ is compact closed.\pause
\item $\FinHilb$ (finite dimensional Hilbert spaces) is compact closed as well.\pause
\item What do we gain if we equip vector spaces with an inner product?\pause
\item We gain the Riesz representation theorem, which means that every object $V$ of
$\FinHilb$ is equipped with a canonical isomorphism $V\to V^*$.\pause
\item On categorical level, every morphism $f\colon V\to U$ is equipped with
another morphism $f^\dag\colon U\to V$ such that $f^{\dag\dag}=f$.
\end{itemize}
\end{frame}
 
\begin{frame}
A {\em dagger category} is a category $\C$ 
equipped with an functor $\dag\colon\C\to\C^{op}$ that is identity
on objects and satisfies $f^{\dag\dag}=f$ for every morphism $f$ of $\C$. In
fact, the $\dag$ functor can be characterized a mapping on the class of morphisms
of $\C$ that has the following properties:
\begin{itemize}
\item $(\id_H)^\dag=\id_H$
\item $(f\circ g )^\dag=g^\dag\circ f^\dag$
\item $f^{\dag\dag}=f$
\end{itemize}
\end{frame}
\begin{frame}
\begin{itemize}
\item $\RelPos$ is probably {\em not} a dagger category.
\item However, there is a partial solution:\pause ~we can replace $\RelPos$ with a
category of self-dual posets with a fixed isomorphism $'\colon A\to A^*$.
\end{itemize}

\end{frame}

\begin{frame}
An {\em involution} on a poset $P$ is mapping $'\colon P\to P$ satisfying the following
conditions.
\begin{itemize}
\item For all $x,y\in P$, $x≤y$ if and only if $y'≤x'$.
\item For all $x\in P$, $x''=x$.
\end{itemize}
A poset equipped with an involution is called {\em involutive poset}, or
{\em poset with involution}.
\end{frame}
\begin{frame}
The category $\RelPosInv$ has posets equipped with involutions for objects and
monotone relations for morphisms. Note that the morphism in $\RelPosInv$ do not
interact with the involutive structure at all. However, the involutive structure
on objects allows us to flip the morphisms: if $f\colon A\sto B$ is a monotone
relation, then there is a monotone relation $f^\dag\colon B\sto A$ given by the
rule
$$
f^\dag(b,a)=f(a',b').
$$ 
It is easy to check that $f^\dag$ is a monotone relation.
Moreover, $\RelPosInv$ equipped with $\dag$ is a dagger category.
\end{frame}
\begin{frame}
\begin{theorem}[GJ]
$\RelPosInv$ is a dagger compact category.
\end{theorem}
\end{frame}
\begin{frame}
A {\em Frobenius structure} in a symmetric monoidal category $(\C,\otimes,I)$ is an object
$A$ equipped with a monoid structure $(A,\nabla,e)$ and a comonoid structure
$(A,\Delta,c)$ such that the following diagram commutes
\begin{equation}
\label{diag:frobenius}
\xymatrix{
A\otimes A
	\ar[rr]^-{\id_A\otimes\Delta}
	\ar[rd]^-{\nabla}
	\ar[dd]_-{\Delta\otimes\id_A}
&
~
&
A\otimes A\otimes A
	\ar[dd]^-{\nabla\otimes\id_A}
\\
~
&
A
	\ar[rd]^-{\Delta}
\\
A\otimes A\otimes A
	\ar[rr]_-{\id_A\otimes\nabla}
&
~
&
A\otimes A
}
\end{equation}
A Frobenius structure is a {\em dagger Frobenius structure} if
$\nabla=\Delta^\dag$ and $m=c^\dag$. Clearly, every dagger Frobenius structure
is completely determined by its (co)monoid structure.
\end{frame}
\begin{frame}
\begin{theorem}[Vicary] Dagger Frobenius structures in $\FinHilb$ are $H^*$-algebras.
\end{theorem}
\end{frame}

\begin{frame}
\begin{problem}
What are dagger Frobenius structures in $\RelPosInv$?
\end{problem}
I do not know, but I have nice examples!
\end{frame}

\begin{frame}
An {\em effect algebra} 
is a partial algebra $(E,\oplus,0,1)$ with a binary 
partial operation $\oplus$ and two nullary operations $0,1$ satisfying
the following conditions.
\begin{enumerate}
\item[(E1)]If $a\oplus b$ is defined, then $b\oplus a$ is defined and
		$a\oplus b=b\oplus a$.
\item[(E2)]If $a\oplus b$ and $(a\oplus b)\oplus c$ are defined, then
		$b\oplus c$ and $a\oplus(b\oplus c)$ are defined and
		$(a\oplus b)\oplus c=a\oplus(b\oplus c)$.
\item[(E3)]For every $a\in E$ there is a unique $a'\in E$ such that
		$a\oplus a'$ exists and $a\oplus a'=1$.
\item[(E4)]If $a\oplus 1$ is defined, then $a=0$.
\end{enumerate}
\end{frame}

\begin{frame}
For every effect algebra $E$, there is a (clearly monotone) relation
$\Delta\colon E\sto E\otimes E$ given by the rule
$$
\Delta(x,(a,b))\Leftrightarrow x\geq a\oplus b
$$ Moreover, there is a monotone relation $c\colon E\sto I$ given by 
$c=E\to I$ (the total relation).
\end{frame}
\begin{frame}
\begin{theorem}[GJ]
For every effect algebra $E$, $(E,\Delta,c)$ is a comonoid that
gives rise to a dagger Frobenius structure on $E$.
\end{theorem}

\end{frame}
\begin{frame}
\begin{center}
\includegraphics[scale=0.4]{Thats_all_folks.svg.png}
\end{center}

\end{frame}

\end{document}
