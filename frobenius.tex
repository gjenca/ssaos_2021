\documentclass{article}
\def\documentlanguage{english}
\usepackage{ifpdf}
\usepackage{ifxetex}
\ifpdf %pdftex
\usepackage[utf8]{inputenc}
\usepackage[T1]{fontenc}
\usepackage[all,pdf,2cell]{xy}\UseAllTwocells\SilentMatrices
\fi
\ifxetex %xetex
\usepackage[all,pdf,2cell]{xy}\UseAllTwocells\SilentMatrices
\fi
\usepackage{amsthm}
\usepackage{amsmath}
\usepackage{amsfonts}
\usepackage{mathtools}
\usepackage{todonotes}
\usepackage{newunicodechar}
\usepackage{cleveref}
\newtheorem{theorem}{Theorem}[section]
\newtheorem{lemma}[theorem]{Lemma}
\newtheorem{proposition}[theorem]{Proposition}
\theoremstyle{definition}
\newtheorem{definition}[theorem]{Definition}
\newtheorem{example}[theorem]{Example}

\newcommand{\cog}{\mathrm{cog}}
\newcommand{\id}{\mathrm{id}}
\newcommand{\C}{\mathcal{C}}
\newcommand{\sar}[2]{\ar[#1]|-@{|}^-{#2}}
\newcommand{\ladj}[1]{{#1}_\diamond}

\newunicodechar{≤}{\leq}
\newunicodechar{≥}{\geq}


\newcommand{\newcategory}[1]{\expandafter\newcommand\csname #1\endcsname{\mathbf{#1}}}
\newcommand{\sto}{\xslashedrightarrow{\hskip 1pc}}

\newcommand{\reldesc}[3]{\ar@{|->}[#1]^-{#2}_-{#3}}
\newcategory{RelPos}
\newcategory{Sup}
\newcategory{Pos}
\newcategory{RelPosInv}

\makeatletter
\def\slashedarrowfill@#1#2#3#4#5{%
  $\m@th\thickmuskip0mu\medmuskip\thickmuskip\thinmuskip\thickmuskip
  \relax#5#1\mkern-7mu%
  \cleaders\hbox{$#5\mkern-2mu#2\mkern-2mu$}\hfill
  \mathclap{#3}\mathclap{#2}%
  \cleaders\hbox{$#5\mkern-2mu#2\mkern-2mu$}\hfill
  \mkern-7mu#4$%
}
\def\rightslashedarrowfill@{%
  \slashedarrowfill@\relbar\relbar\mapstochar\rightarrow}
\newcommand\xslashedrightarrow[2][]{%
  \ext@arrow 0055{\rightslashedarrowfill@}{#1}{#2}}
\makeatother
\title{Effect algebras as Frobenius monoids}
\author{Gejza Jenča}
\begin{document}
\maketitle
\section{Monotone relations}

We write $|A|$ for the underlying set of a poset $A$. If $A,B$ are sets
and $f\subseteq A\times B$ is a relation between them, we denote $(a,b)\in f$ by
$f(a,b)$ of by $a\xmapsto{f} b$.
\begin{definition}
Let $P,Q$ be posets. A relation $f\subseteq |P|\times |Q|$ is {\em monotone relation
from $P$ to $Q$}
if and only if, for every $p_1,p_2\in P$ and $q_1,q_2\in Q$,
$$
p_1\geq p_2\text{ and } f(p_2,q_1)\text{ and }q_1\geq q_2\text{ imply } f(p_1,q_2)
$$
\end{definition}
We write $f\colon P\sto Q$ for a monotone relation $f$ from $P$ to $Q$.
\begin{example}
Let $P$, $Q$ be posets. Both the universal 
relation $P\times Q\subseteq P\times Q$ and the empty relation $\emptyset\subseteq P\times
Q$ are monotone.
\end{example}

Every monotone mapping gives rise to a monotone relation:
\begin{proposition}
For every monotone mapping $f\colon P\to Q$, there is a monotone relation
$\ladj f\colon P\sto Q$ given by the rule $\ladj f(p,q)$ if and only if
$f(p)\geq q$.
\end{proposition}
\begin{proof}
Suppose that 
$p_1,p_2\in P$ and $q_1,q_2\in Q$, $p_1\geq p_2$, $\ladj f(p_2,q_1)$ and $q_1\geq
q_2$. Then 
$f(p_1)\geq f(p_2)\geq q_1\geq q_2$, hence $f(p_1)\geq q_2$, that means, $\ladj
f(p_1,q_2)$.
\end{proof}

It is instructive to visualize a monotone relation between two disjoint finite
posets $P$ and $Q$ in the following way.
\begin{itemize}
\item Draw the Hasse diagram of $Q$.
\item Draw the Hasse diagram of $P$ over the diagram of $Q$.
\item Draw some additional lines between elements of $P$ and elements of $Q$,
so that the resulting picture is a Hasse diagram of a poset.
\end{itemize}
Denote the resulting poset with the underlying
set $|P|\cup|Q|$ by $C$.
This poset then determines a monotone relation $f_C\subseteq P\times Q$ 
given by the rule $f_C(p,q)$ if and only if $q\leq_C p$.
\begin{figure}
\label{fig:PQ}
\begin{center}
\includegraphics{PQ}
\includegraphics{cograph}
\end{center}
\end{figure}

On the other hand, whenever $f$ is a monotone relation from $P$ to $Q$,
there is a poset $\cog(f)$, called {\em cograph of $f$} determined by $f$, as follows.
\begin{itemize}
\item The underlying set of $\cog(f)$ is the disjoint union $|P|\sqcup|Q|$.
\item For $x,y\in\cog(f)$, $x\leq y$ if and only if one of the following is true:
\begin{itemize}
\item $x,y\in P$ and $x\leq_P y$
\item $x,y\in Q$ and $x\leq_Q y$
\item $x\in P$ and $y\in Q$ and $f(x,y)$.
\end{itemize}
\end{itemize}
It is easy to check that $\cog(f)$ is a poset.

The above constructions are mutually inverse: if $f$ is a monotone relation, we may
extract $f$ back from its cograph and vice versa.

\begin{example}
Figure \ref{fig:PQ} shows two finite posets $P,Q$ and a cograph of a monotone
relation $r\colon P\sto Q$, where
$$
r=(\{a,c\}\times\{d,e,f\})\cup(\{b\}\times\{d,f\}).
$$
The ``additional lines'' in $\cog(r)$ are red-colored.
\end{example}

Let $P,Q,R$ be posets, let $f\colon P\sto Q$ and $g\colon Q\sto R$ be
monotone relations. The composite relation $g\circ f\subseteq|P|\times|R|$ is given
by the rule
$$
(g\circ f)(p,r)\text{ if and only if }
	f(p,q)\text{ and }g(q,r)\text{ for some }q\in Q.
$$
It is easy to check that the composite relation of two monotone relations is
monotone and that the opration of composition is associative.

One can visualise the notion of composition of monotone relations using cographs.
For a pair of monotone relations $f\colon P\sto Q$ and $g\colon Q\sto R$ between
finite posets, we can draw their ``merged cograph'' consisting of 
$\cog(f)$ and $\cog(g)$ in the same picture. If we delete the elements of the
middle poset $Q$, while retainging the comparability information between elements
of $P$ and elements of $R$, we obtain the Hasse diagram of the $\cog(g\circ f)$.
....

For a poset $P$, the {\em identity monotone relation} is the relation
$\id_P\colon P\sto P$ given by the rule
$$
\id_P(x,y)\Leftrightarrow x\geq y
$$
It is easy to see that for every monotone relation $f\colon P\sto Q$,
$f=\id_Q\circ f=f\circ\id_P$.

The category of posets and monotone relations, denoted by $\RelPos$, is a category
whose objects are posets and arrows are monotone relations.

Let us put the category $\RelPos$ into some context: let $T\colon\Pos\to\Pos$ be the {\em lowerset monad} on the category of posets,
defined as follows. On objects, $T(A)$ is the poset of all lower sets (or order
ideals) of a poset $A$. For a monotone mapping $f\colon A\to B$, the
morphism $T(f)\colon A\to B$ is given by the rule
$$
T(f)(X)=\{y\in B\mid y\leq f(x)\text{ for some }x\in X\}
$$
For a poset $A$, the unit $\eta_A\colon A\to T(A)$ of the lowerset monad is
given by the rule 
$$
\eta_A(a)=a^\downarrow=\{x\in A\mid x\leq a\}
$$
The multiplication $\mu_A\colon T^2(A)\to T(A)$ of the lowerset monad is given
by the rule 
$$
\mu_A(\mathbb X)=\bigcup_{X\in\mathbb X}X
$$
where $\mathbb X$ is a ``lowerset of lowersets''.

The Eilenberg-Moore category $\Pos^T$ for
this monad is equivalent to the category $\Sup$ of complete-join semilattices.
On the other hand, it is easy to check that $\RelPos$ is equivalent (even
isomorphic) to the Kleisli category for $T$. Indeed, every isotone relation
$f\colon P\sto Q$ is exactly the same thing as an isotone mapping $f\colon P\to
T(Q)$ -- a morphism in the Kleisli category $\Pos_T$. Hence there is adjunction
between the category $\Pos$ and the category $\RelPos$. The left adjoint functor
of this adjunction is identity on objects and takes every isotone map $f$ to $\ladj
f$. The right adjoint functor
$G\colon\RelPos\to\Pos$ of this adjunction takes every object $A$ of 
$\RelPos$ to the set of all lowersets of $A$. On morphisms,
$$
G(f)(X)=\{y\in B\mid f(x,y)\text{ for some }x\in X\}
$$
for a morphism $f\colon A\sto B$ in $\RelPos$.

 
If $P,Q$ are posets, then their disjoint union $P\sqcup Q$ is both the product and
the coproduct in $\RelPos$. However, $\RelPos$ lacks some equalizers, as the
following example shows.
\begin{figure}
\label{fig:fg}
\begin{center}
\includegraphics{fg}
\end{center}
\caption{The relations $f$ and $g$ from \Cref{ex:fg}}
\end{figure}
\begin{example}
\label{ex:fg}
Let $A=\{1,2,3\}$ and $B=\{a,b,c\}$ be two antichains, so that every relation
between them is monotone. Define $f,g\colon A\sto B$ as in \ref{fig:fg}. We shall
prove that that the pair $f,g$ has no equalizer in $\RelPos$. Suppose that
$X$ is the equalizer of $f,g$ in $\RelPos$.
Since
$G\colon\RelPos\to\Pos$ is a right adjoint functor, $G$ preserves limits. Hence the
equalizer $E$ in $\Pos$
$$
\xymatrix{
E
	\ar[r]^-{e}
&
G(A)
	\ar@<.5ex>[r]^-{G(f)}
	\ar@<-.5ex>[r]_-{G(g)}
&
G(B)
}
$$
must be isomorphic to $G(X)$. In particular, $E$ must be a distributive lattice.
However, an easy computation shows that $E$ is isomorphic to
the poset 
$$
\{\emptyset,\{1\},\{1,2\},\{2,3\},\{1,2,3\}\}
$$
ordered by inclusion, which is not a distributive lattice.
\end{example}

\section{$\RelPos$ is a compact closed category}

The category $\Pos$ has all products, the product being the usual direct product
of posets. We can transfer
the cartesian monoidal structure to a (non-cartesian) monoidal structure
on $\RelPos$, as follows.
The direct product $\otimes$ of posets is a bifunctor from
$\RelPos\times\RelPos$ to $\RelPos$. Indeed, for a pair of monotone relations
$f\colon A\sto B$ and $g\colon C\sto D$ the isotone relation $(f\otimes g)\colon
A\otimes C\sto B\otimes D$ by the rule
$$
(f\otimes g)((a,c),(b,d)) \Leftrightarrow f(a,b)\text{ and }g(c,d)
$$


As $\otimes$ is the product in $\Pos$ and $\Pos$ is isomorphic to a subcategory of $\RelPos$,
$\otimes$ satisfies the coherence conditions for a symmetric monoidal category
\cite[Chapter VII]{mac1998categories}, so $(\RelPos,\otimes,1)$ is a symmetric
monoidal category, we just need to replace the natural isomorphisms
$\alpha,\lambda,\rho,\sigma$ by their $\RelPos$ versions
$\ladj\alpha,\ladj\lambda,\ladj\rho,\ladj\sigma$. For example, for a poset $A$,
we obtain $\ladj\rho\colon A\otimes 1\sto A$ by the rule 
$$
\ladj\rho((a_2,\bullet),a_1)\Leftrightarrow \rho(a_2,\bullet)\geq
a_1\Leftrightarrow a_2\geq a_1
$$
and similarly for the other isomorphisms. Note that the inverse morphism of
$\ladj\rho$ is given by
$$
\ladj\rho^{-1}(a_1,(a_0,\bullet))\Leftrightarrow a_1\geq a_0
$$ 

For a poset $A$, we write $A^*$ for the {\em dual poset}, with the same underlying
set and the opposite partial order. For every monotone relation $f\colon A\sto B$
there is a monotone relation $f\colon B^*\sto A^*$, given by the rule
$f^*(b,a)\Leftrightarrow f(a,b)$. Note that $\cog(f^*)\simeq (\cog(f))^*$.

A {\em dual object} to an object $A$ of a symmetric monoidal category $(\C,\otimes,I)$ is an object
$A^*$ such that there are morphisms
$\eta_A\colon I\to A^*\otimes A$ and $\epsilon_A\colon A\otimes A^*\to I$ such
that the diagrams
\begin{equation}
\label{diag:compact}
\xymatrix{
A
	\ar[r]^{\rho_A^{-1}}
	\ar[dd]_{\id_A}
&
A\otimes I
	\ar[d]^{\id_A\otimes\eta_A}
\\
~
&
A\otimes A^*\otimes A
	\ar[d]^{\epsilon_A\otimes\id_A}
\\
A
&
I\otimes A
	\ar[l]^{\lambda_A}
}
\qquad
\xymatrix{
A^*
	\ar[r]^{\lambda_{A^*}^{-1}}
	\ar[dd]_{\id_{A*}}
&
I\otimes A^*
	\ar[d]^{\eta_A\otimes\id_{A^*}}
\\
~
&
A^*\otimes A\otimes A^*
	\ar[d]^{\id_{A^*}\otimes\epsilon_A}
\\
A
&
A^*\otimes I
	\ar[l]^{\rho_{A^*}}
}
\end{equation}
commute. The morphisms $\eta_A$ and $\epsilon_A$ are called {\em coevaluation} and
{\em evaluation}, respectively. 

A symmetric monoidal category is {\em compact closed}
if every of its objects has a dual. For every object $A$ of a compact closed
category, every dual of $A$ is isomorphic to every other dual of $A$.
For every morphism $f\colon A\to B$ in
a compact closed category, there is a dual morphism $f^*\colon B^*\to A^*$
given by the composite
$$
\xymatrix{
B^*
	\ar[r]^{\lambda_{B^*}^{-1}}
&
I\otimes A^*
	\ar[r]^{\eta_A\otimes\id_{B^*}}
&
A^*\otimes A\otimes B^*
	\ar[r]^{\id_{A^*}\otimes f\otimes\id_{B^*}}
&
A^*\otimes B\otimes B^*
	\ar[r]^{{\id_A^*}\otimes\epsilon_B}
&
A^*\otimes I
	\ar[r]^{\rho_{A^*}}
&
A^*
}
$$

\begin{theorem}
$(\RelPos,\times,\mathbf 1)$ is a compact closed category.
\end{theorem}
\begin{proof}
For every poset $P$, the dual object $P^*$ is the dual poset, 
the coevaluation relation $\eta_P\colon\mathbf 1\to P^*\otimes P$
and the evaluation relation $\varepsilon_P\colon P\otimes P^*$ are given by
the rules
\begin{align*}
\bullet\xmapsto{\eta_P}(x,y)&\Leftrightarrow x\geq y\\
(x,y)\xmapsto{\varepsilon_P}\bullet&\Leftrightarrow x\geq y
\end{align*}
To prove that the left-hand side diagram in \eqref{diag:compact} commutes
means to prove that the monotone relation
\begin{equation}
\label{eq:composite}
\xymatrix{
P
	\sar{r}{\rho^{-1}_P}
&
P\otimes 1
	\sar{r}{\id_P\otimes\eta_P}
&
P\otimes P^*\otimes P
	\sar{r}{\varepsilon_P\otimes\id_P}
&
1\otimes P
	\sar{r}{\lambda_P}
&
P
}
\end{equation}
is equal to $\id_P$. For $x,z\in P$, we have $\id_P(x,z)$ if and only if
$x\geq z$. 

Suppose that $(x,z)$ are in the \eqref{eq:composite} relation, that means,
there exist $x_1,x_2,y,z_1,z_2\in P$ such that
\begin{itemize}
\item $\rho_P^{-1}(x,(x_2,\bullet))$ meaning that $x\geq x_2$;
\item $(\id_P\otimes\eta_P)((x_2,\bullet),(x_1,y,z_2))$ meaning that $x_2\geq x_1$
and $y\geq z_2$;
\item $(\epsilon_P\otimes\id_P)((x_1,y,z_2),(\bullet,z_1))$ meaning that $x_1\geq
y$ amd $z_2\geq z_1$;
\item $\lambda_P((\bullet,z_1),z)$ meaning that $z_1\geq z$. 
\end{itemize}
This implies that
$$
x\geq x_2\geq x_1\geq y\geq z_2\geq z_1\geq z,
$$
so $x\geq z$ meaning that $\id_P(x,z)$.
For the opposite implication,
we may put all of the $x_2,x_1,y,z_2,z_1$ to be equal to $x$. 
The proof of the other condition in the definition of a dual object
is similar and is thus omitted.
\end{proof}

Let us introduce a notation, that will allow us to simplify our proofs.
We will express the fact that $(x,z)$ are in the composite relation
\eqref{eq:composite} like this:
\begin{equation}
\label{eq:compositegraph}
\xymatrix@C=3pc{
x
	\ar@{|->}[r]^{\rho^{-1}_P}_{x\geq x_2}
&
(x_1,\bullet)
	\ar@{|->}[r]^-{\id_P\otimes\eta_P}_{\substack{x_2\geq x_1\\y\geq z_2}}
&
(x_2,y,z)
	\ar@{|->}[r]^-{\varepsilon_P\otimes\id_P}_{\substack{x_1\geq y\\z_2\geq z_1}}
&
(\bullet,z_1)
	\ar@{|->}[r]^-{\lambda_P}_-{z_1\geq z}
&
z
}
\end{equation}
Every arrow in \eqref{eq:compositegraph} has the name of the relation
over it and the conditions characterizing the relation are stacked below. 
Every variable occuring in \eqref{eq:compositegraph} is bounded by an
existential quantifier, except for the leftmost and the rightmost variables $x,z$.

\section{Dagger compact structure on $\RelPosInv$}
A {\em dagger category} is a category $\C$ 
equipped with an functor $\dag\colon\C\to\C^{op}$ that is identity
on objects and satisfies $f^{\dag\dag}=f$ for every morphism $f$ of $\C$. In
fact, the $\dag$ functor can be characterized a mapping on the class of morphisms
of $\C$ that has the following properties:
\begin{itemize}
\item $(\id_H)^\dag=\id_H$
\item $(f\circ g )^\dag=g^\dag\circ f^\dag$
\item $f^{\dag\dag}=f$
\end{itemize}

An {\em involution} on a poset $P$ is mapping $'\colon P\to P$ satisfying the following
conditions.
\begin{itemize}
\item For all $x,y\in P$, $x≤y$ if and only if $y'≤x'$.
\item For all $x\in P$, $x''=x$.
\end{itemize}
A poset equipped with an involution is called {\em involutive poset}, or
{\em pose with involution}.

The category $\RelPosInv$ has posets equipped with involutions for objects and
monotone relations for morphisms. Note that the morphism in $\RelPosInv$ do not
interact with the involutive structure at all. However, the involutive structure
on objects allows us to flip the morphisms: if $f\colon A\sto B$ is a monotone
relation, then there is a monotone relation $f^\dag\colon B\sto A$ given by the
rule
$$
f^\dag(b,a)=f(a',b').
$$ 
It is easy to check that $f^\dag$ is a monotone relation.
Moreover, $\RelPosInv$ equipped with $\dag$ is a dagger category.

For a pair of posets with involution $A,B$, there is an involution
on $A\otimes B$ given by the rule $(a,b)'=(a',b')$, for all $(a,b)\in A\otimes B$.
It is obvious that this gives us a monoidal structure on $\RelPosInv$.
Moreover, every involution on a poset $A$ is at the same time an involution on is
dual $A^*$. Again, it is clear that the axioms \eqref{diag:compact} are satisfied,
so $\RelPosInv$ is a compact closed category.

\begin{definition}
\label{def:dagsymmonoidal}
Let $(\C,\otimes,I)$ be a symmetric monoidal category that is at the same time
a dagger category. We say that $\C$ is a {\em dagger symmetric monoidal category}
if the following equalities are satisfied, for all objects $A,B,C,D$ and
morphisms $f\colon A\to B$ and $g\colon C\to D$.
\begin{itemize}
\item $(f\otimes g)^\dag=f^\dag\otimes g^\dag$
\item $\alpha_{A,B,C}^{-1}=\alpha_{A,B,C}^\dag$
\item $\lambda_A^{-1}=\lambda_A^\dag$
\item $\rho_A^{-1}=\rho_A^\dag$
\item $\sigma_{A,B}^{-1}=\sigma_{A,B}^\dag$
\end{itemize}
\end{definition}


\begin{definition}
A dagger symmetric monoidal category $(\C,\otimes,I)$ is a {\em dagger compact category} if
 $\C$ is a compact closed category, $\C$ is a dagger category, and
the following diagram commutes
\begin{equation}
\label{diag:dagcompact}
\xymatrix{
I
	\ar[r]^-{\epsilon_A^{\dag}}
	\ar[rd]_{\eta_A}
&
A\otimes A^*
	\ar[d]^{\sigma_{A,A^*}}
\\
~
&
A^*\otimes A
}
\end{equation}

\end{definition}
\begin{theorem}
$\RelPosInv$ is a dagger compact category.
\end{theorem}
\begin{proof}

We need to check that $\RelPosInv$ is a dagger symmetric monoidal category. Let 
$f\colon A\sto C$ and $g\colon B\sto D$ be morphisms
in $\RelPosInv$.
 For all $(a,b)\in A\otimes B$ and
$(c,d)\in C\otimes D$,
\begin{multline*}
(f\otimes g)^\dag((b,d),(a,c))\Leftrightarrow
(f\otimes g)((a,c)',(b,d)')\Leftrightarrow
(f\otimes g)((a',c'),(b',d'))\Leftrightarrow\\
f(a',b')\wedge g(c',d')\Leftrightarrow
f^\dag(b,a)\wedge g^\dag(d,c)\Leftrightarrow
(f^\dag\otimes g^\dag)((b,d),(a,c))
\end{multline*}
To prove that $\ladj\rho^{-1}=\ladj\rho^\dag$, let $A$ be an object of $\RelPosInv$
and let $a_1,a_0\in A$. 

Then
$$
{\rho_A}^\dag(a_1,(a_0,\bullet))\Leftrightarrow\rho_A((a_0',\bullet),a_1')
\Leftrightarrow a_0'\geq a_1'\Leftrightarrow a_1\geq a_0\Leftrightarrow
\rho_A^{-1}(a_1,(a_0,\bullet))
$$
hence $\rho_A^\dag=\rho_A^{-1}$.
The remaining conditions of \Cref{def:dagsymmonoidal}
can be proved in a similar way.

To check that the diagram \eqref{diag:dagcompact} commutes, let
$(w,z)\in A\otimes A^*$ and compute
$$
\epsilon_A^\dag(\bullet,(w,z))\Leftrightarrow\epsilon_A((w,z)',\bullet)\Leftrightarrow\epsilon_A((w',z'),\bullet)
\Leftrightarrow w'\geq z'
$$
so
$$
\xymatrix{
\bullet
	\reldesc{r}{\epsilon_A^\dag}{w'\geq_A z'}
&
(w,z)
	\reldesc{r}{\sigma_{A,A^*}}{\substack{{w\geq_A y}\\{z\geq_{A^*} x}}}
&
(x,y)
}
$$
meaning that
$$
x\geq_A z\geq_A w\geq_A y,
$$
hence
$$
\xymatrix {
\bullet
	\reldesc{r}{\eta_A}{x\geq_A y}
&
(x,y)
}
$$
For the opposite implication, it suffices to put $z=x$ and $w=y$.
\end{proof}

\section{Effect algebras}

An {\em effect algebra} 
is a partial algebra $(E,\oplus,0,1)$ with a binary 
partial operation $\oplus$ and two nullary operations $0,1$ satisfying
the following conditions.
\begin{enumerate}
\item[(E1)]If $a\oplus b$ is defined, then $b\oplus a$ is defined and
		$a\oplus b=b\oplus a$.
\item[(E2)]If $a\oplus b$ and $(a\oplus b)\oplus c$ are defined, then
		$b\oplus c$ and $a\oplus(b\oplus c)$ are defined and
		$(a\oplus b)\oplus c=a\oplus(b\oplus c)$.
\item[(E3)]For every $a\in E$ there is a unique $a'\in E$ such that
		$a\oplus a'$ exists and $a\oplus a'=1$.
\item[(E4)]If $a\oplus 1$ is defined, then $a=0$.
\end{enumerate}

Effect algebras were introduced by Foulis and Bennett in their paper 
\cite{FouBen:EAaUQL}.

In an effect algebra $E$, we write $a\leq b$ if and only if there is
$c\in E$ such that $a\oplus c=b$.  It is easy to check that for every effect
algebra $E$, $\leq$ is a partial order on $E$ and $a\mapsto a'$
is an involution on this poset.  Moreover, it is possible to introduce
a new partial operation $\ominus$; $b\ominus a$ is defined if and only if $a\leq
b$ and then $a\oplus(b\ominus a)=b$.  It can be proved that, in an effect
algebra, $a\oplus b$ is defined if and only if $a\leq b'$ if and only if $b\leq
a'$.

A {\em Frobenius structure} in a symmetric monoidal category $(\C,\otimes,I)$ is an object
$A$ equipped with a monoid structure $(A,\nabla,e)$ and a comonoid structure
$(A,\Delta,c)$ such that the following diagram commutes
\begin{equation}
\label{diag:frobenius}
\xymatrix{
A\otimes A
	\ar[rr]^-{\id_A\otimes\Delta}
	\ar[rd]^-{\nabla}
	\ar[dd]_-{\Delta\otimes\id_A}
&
~
&
A\otimes A\otimes A
	\ar[dd]^-{\nabla\otimes\id_A}
\\
~
&
A
	\ar[rd]^-{\Delta}
\\
A\otimes A\otimes A
	\ar[rr]_-{\id_A\otimes\nabla}
&
~
&
A\otimes A
}
\end{equation}

A Frobenius structure is a {\em dagger Frobenius structure} if
$\nabla=\Delta^\dag$ and $m=c^\dag$. Clearly, every dagger Frobenius structure
is completely determined by its (co)monoid structure.
We say that a dagger Frobenius structure is {\em commutative}
if $\nabla$ is commutative. Note that $\nabla$ is commutative if and only if
$\Delta$ is cocommutative.

For every effect algebra $E$, there is a (clearly monotone) relation
$\Delta\colon E\sto E\otimes E$ given by the rule
$$
\Delta(x,(a,b))\Leftrightarrow x\geq a\oplus b
$$. Moreover, there is a monotone relation $c\colon E\sto I$ given by 
$c=E\to I$ (the total relation).
\begin{theorem}
With these operations, every effect algebra is a dagger commutative Frobenius
structure.
\end{theorem}
\begin{proof}
Let $E$ be an effect algebra. Clearly, $\Delta$ is cocommutative. 
Let us prove that $\Delta$ is coassociative. This amounts to
prove that the diagram
$$
\xymatrix{
E
	\ar[r]^{\Delta}
	\ar[d]_{\Delta}
&
E\otimes E
	\ar[d]^{\Delta\otimes\id_E}
\\
E\otimes E
	\ar[r]_{\id_E\otimes\Delta}
&
E\otimes E\otimes E
}
$$
\end{proof}
commutes. For every $x\in E$ and $(a_1,b_1,c_1)\in E\otimes E\otimes E$, the relation $(\Delta\otimes\id_E)\circ\Delta$
is given by
$$
\xymatrix{
x
	\ar@{|->}[r]^-{\Delta}_-{x\geq y\oplus c_2}
&
(y,c_2)
	\ar@{|->}[r]^-{\Delta\otimes\id_E}_{\substack{y\geq a_1\oplus
b_1\\c_2\geq c_1}}
&
(a_1,b_1,c_1)
}
$$
We claim that this is equivalent to $x\geq a_1\oplus b_1\oplus c_1$. Indeed,
if $x\geq a_1\oplus b_1\oplus c_1$, then we may put $y:=a_1\oplus b_1$ and
$c_2:=c_1$. On the other hand, $x\geq y\oplus c_2$ and $y\geq a_1\oplus b_1$ imply
that $a_1\oplus b_1\oplus c_2$ exists and $x\geq a_1\oplus b_1\oplus c_2$. From
this and $c_2\geq c_1$ we obtain $x\geq a_1\oplus b_1\oplus c_1$. The rest follows
by a symmetry.

To prove that $c\colon E\to I$ is a counit of $\Delta$ means to prove that
the diagram
$$
\xymatrix{
~
&
E
	\ar[ld]_-{\rho_E^{-1}}
	\ar[d]^{\Delta}
	\ar[rd]^{\lambda_E^{-1}}
\\
E\otimes I
&
E\otimes E
	\ar[l]^-{\id_E\otimes c}
	\ar[r]_-{c\otimes\id_E}
&
I\otimes E
}
$$
commutes.
For every $x,a_1\in E$ the relation $(\id_E\otimes c)\circ\nabla$ amounts to
\begin{equation}
\label{diag:counit}
\xymatrix {
x
	\ar@{|->}[r]^-{\Delta}_-{x\geq a_2\oplus b}
&
(a_2,b)
	\ar@{|->}[r]^-{\id_E\otimes c}_{a_2\geq a_1}
&
(a_1,\bullet)
}
\end{equation}
which clarly implies $x\geq a_1$, that means, $\rho_E^{-1}(x,a_1)$. For the
opposite implication, it suffices to put $a_2=x$ and $b=0$ and we see that
the left-hand triangle in \eqref{diag:counit} commutes. The commutativity
of the other triangle can be proved similarly.

We now know that $(E,\Delta,c)$ is cocommutative comonoid, hence $(E,\nabla,e)$
where $\nabla=\Delta^\dag$ and $e=c^\dag$ is a commutative monoid. 

Before we prove that this monoid-comonoid pair gives rise to a Frobenius structure,
let us describe the $\nabla$ and $e$ in an explicit way.

For every $a,b,x\in E$, we have 
$\nabla((a,b),x)\Leftrightarrow\Delta^\dag((a,b),x)\Leftrightarrow\Delta(x',(a',b'))$
meaning that $x'\geq a'\oplus b'$. The existence of $a'\oplus b'$ is equivalent
to $a''=a\geq b'$. Moreover, $x'\geq a'\oplus b'$ is equivalent to
$(a'\oplus b')'\geq x$ and, by ...,
$$
(a'\oplus b')'=a\odot b,
$$
so the $\nabla$ relation can be characterized by
$$
\nabla((a,b),x)\Leftrightarrow a\odot b\geq x.
$$

Let us prove that the lower triangle of \eqref{diag:frobenius} commutes. For every
$(a,b),(c,d)\in E\otimes E$, $(\Delta\circ\nabla)((a,b),(c,d))$ means that
\begin{equation*}
\xymatrix{
(a,b)
	\ar@{|->}[r]^-\nabla_{a\odot b\geq x}
&
x
	\ar@{|->}[r]^-\Delta_{x\geq c\oplus d}
&
(c,d)
}
\end{equation*}
which is clearly equivalent to 
\begin{equation}
\label{diag:frobdiagonal}
a\odot b\geq c\oplus d
\end{equation}
On the other hand, $((\id_E\otimes\nabla)\circ(\Delta\otimes\id_E))((a,b),(c,d))$
means that
\begin{equation}
\label{diag:frobdownright}
\xymatrix{
(a,b)
	\ar@{|->}[r]^-{\Delta\otimes\id_E}_-{\substack{a\geq c_1\oplus y\\b\geq
b_1}}
&
(c_1,y,b_1)
	\ar@{|->}[r]^-{\id_E\otimes\nabla}_-{\substack{c_1\geq c\\y\odot b_1\geq
d}}
&
(c,d)
}
\end{equation}

Suppose \eqref{diag:frobdiagonal}. Note that
$b\geq a\odot b\geq c\oplus d\geq d$, hence $b''=b\geq d$ so $d\oplus b'$
exists. Put $c_1=c$, $b_1=b$ and $y=d\oplus b'$ and note that 
$$
a\ominus b'=a\odot b\geq  c\oplus d
$$
implies that 
$$
a\geq c\oplus d\oplus b'=c_1\oplus y.
$$
Compute
$$
y\odot b_1=(d\oplus b')\odot b=(d\oplus b')\ominus b'=d\geq d
$$
and we have proved \eqref{diag:frobdownright}.

Suppose \eqref{diag:frobdownright}. Note that the existence of
$y\odot b_1$ implies that $b_1\geq y'$ and that the existence of
$c_1\oplus y$ means that $y'\geq c_1$. From this we obtain
\begin{align*}
b\geq b_1\geq y'\geq c_1\geq c\\
b\geq b_1\geq y\odot b_1\geq d
\end{align*}
Since $b\geq b_1$, $b_1'\geq b'$ and $y\geq b_1'\geq b'$. This implies that
$$
y\ominus b'\geq y\ominus b_1'=y\otimes b_1\geq d
$$
hence $y\geq d\oplus b'$. Finally,
$$
a\geq c_1\oplus y\geq c\oplus y\geq c\oplus d\oplus b'
$$
which implies $a\otimes b=a\ominus b'\geq c\oplus d$ and
we have proved \eqref{diag:frobdiagonal}.
\end{document}`	
